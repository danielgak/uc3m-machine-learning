\documentclass[10pt, spanish, pdftex]{../.memo/plantilla_memorias}

%%%%%%%%%%%%%%%%%%%%%%%%%%%%%%%%%%%%%%%%%%%%%%%%%%%%%%%%
% Plantilla de estilos para memorias de prácticas UC3M
%			Universidad Carlos III de Madrid
% Autor: Aitor Alonso Núñez
% Última actualización: 14 de Noviembre de 2016
%%%%%%%%%%%%%%%%%%%%%%%%%%%%%%%%%%%%%%%%%%%%%%%%%%%%%%%%

%%%%% Preámbulo %%%%%
\usepackage{fancyhdr}		% Permite añadir encabezado y pie de página
\usepackage{graphicx}		% Permite añadir imágenes
\usepackage{caption}
\usepackage{subcaption}
\usepackage{lipsum}
\usepackage[hidelinks]{hyperref}	% Ahora los elemento del índice actúan como enlaces, también se permite enlaces a internet
\usepackage{nameref}				% Permite añadir etiquetas \label{nombre} a las que referenciar con \nameref{intro}
\author{Aitor Alonso Núñez}


%%%%% Co-autores y NIA %%%%%
%%%%% Dejar corchetes {} en blanco si es necesario
%%%%% Si el número de coautores está entre 3 y 4 usar estas opciones, en caso contrario comentarlas
%\coautoresNumCuatrotrue
%\coautoresCuatro{Compañero1 Apellido}{NIA 100XXXXXX}{Compañero2 Apellido}{NIA 100XXXXXX}{Compañero3 Apellido}{NIA 100XXXXXX}{Compañero4 Apellido}{NIA 100XXXXXX}

%%%%% Si el número de coautores está entre 0 y 2 usar estas opciones, en caso contrario comentarlas
% Dejar corchetes {} en blanco si es necesario
\coautoresNumCuatrofalse
\coautoresDos{Daniel Gak Anagrov}{NIA 100318133}{Gr. 83}

%%%%% Añadir dirección de email en la portada (solo válido para ti y un compañero) %%%%%
\correofalse
%\correotrue
%\emails{100346169alumnos.uc3m.es}{correo}

%%%%% Datos básicos (titulación, asignatura, título del documento) %%%%%
\datos{Grado en Ingeniería Informática}{Aprendizaje Automático}{Mario AI: Tutorial 3\\Experimentación múltiple}

%%%%% Encabezado %%%%%
\header

%%%%% Pie de Página %%%%%
\footer
\pagestyle{fancy}

\begin{document}
%%%%% Portada %%%%%a
\titleMain

%%%%% Índice %%%%%
\tableofcontents 	% Añadimos índice
\newpage 			% Después del índice, le indicamos iniciar en una página nueva

%%%%% CONTENIDO DEL DOCUMENTO %%%%%
\section{Ejercicio 1: KnowledgeFlow}
\subsection{Ejecutar el flujo de conocimiento. Para ello ejecutar la opción Start loading del nodo \texttt{Arff Loader}. Seleccionar
la opción \texttt{Show Results} en los nodos \texttt{TextViewer}.\\a) ¿Qué se muestra en cada uno de ellos?}
\p{En el nodo \texttt{TextViewer} que está unido al nodo \texttt{J48} se muestra el árbol creado por el clasificador J48.
En el nodo \texttt{TextViewer} que está unido al nodo \texttt{Classifier PerformanceEvaluator} se muestra el porcentaje de instancias
clasificadas correctamente y diversa información adicional como la matriz de confusión.}
\p{Por último, en el nodo \texttt{GraphViewer}
que está unido al nodo \texttt{J48} se muestra una lista con los árboles de decisión que ha generado cada uno de los \textit{folds}
de la \textit{cross validation}. Si hacemos clic sobre un elemento de esta lista nos abre la representación gráfica del árbol
de decisión, con las reglas que se aplican para clasificar en cada nodo.}

\subsection{b) ¿Cuál es el porcentaje de instancias clasificadas correctamente?}
\p{Se clasifican correctamente el 86,2105\% de las instancias (28.071 instancias bien clasificadas).}

\subsection{Guardar el diagrama de flujo en formato .kfml como \texttt{e1-suppliedtestset.kfml}.\\
c) ¿Cuál es la utilidad de crear flujos de conocimiento con esta interfaz de Weka?}
\p{Con esta interfaz podemos visualizar las distintas fases por la que pasa el estudio y filtrado de los datos, y las fases que
producen el aprendizaje. Si no se utilzara esta herramienta, habría que estar repitiendo  
análisis, fitros, preprocesados y construcciones de modelos en el explorador de weka, mientas que al crear diferentes flujos de
conocimiento, defines lineas de trabajo reutilizables permitiéndote constuir un árbol de flujos complejo, pero fácil de gestionar.}

\section{Ejercicio 2: Experimenter}
\p{Tras la ejecución del test sonre los resultado del análisis de los datos se ha obtenido la siguiente salida:}
\begin{table}[h]
	\centering
	\begin{tabular}{l|r|cccccc|}
	\hline
	\multicolumn{1}{|l|}{Dataset}                           & J48     & PART    & ZeroR   & NaiveBayes & IbK1    & IbK3    & IbK7    \\ \hline
	\multicolumn{1}{|l|}{T3BotAgent\_Discr}                 & 100.00  & 100.00  & 28.80 * & 99.40 *    & 99.51 * & 98.79 * & 98.20 * \\
	\multicolumn{1}{|l|}{T3BotAgent\_Discr\_Selec}          & 99.70   & 99.52   & 28.80 * & 98.67 *    & 99.35   & 98.99 * & 98.48 * \\
	\multicolumn{1}{|l|}{T3BotAgent\_Discr\_noObs}          & 100.00  & 100.00  & 28.80 * & 99.49 *    & 99.79   & 99.52 * & 99.55 * \\
	\multicolumn{1}{|l|}{T3BotAgent\_Discr\_noObs\_noEva}   & 100.00  & 100.00  & 28.80 * & 99.00 *    & 99.96   & 99.42 * & 99.14 * \\
	\multicolumn{1}{|l|}{T3HumanAgent\_Discr}               & 99.90   & 99.90   & 25.00 * & 98.89 *    & 99.20 * & 98.48 * & 96.68 * \\
	\multicolumn{1}{|l|}{T3HumanAgent\_Discr\_Selec}        & 99.40   & 99.40   & 25.00 * & 98.55 *    & 99.05   & 98.67   & 98.23 * \\
	\multicolumn{1}{|l|}{T3HumanAgent\_Discr\_noObs}        & 99.90   & 99.90   & 25.00 * & 99.20 *    & 99.30 * & 99.55   & 99.40   \\
	\multicolumn{1}{|l|}{T3HumanAgent\_Discr\_noObs\_noEva} & 99.90   & 99.90   & 25.00 * & 99.89      & 99.39   & 99.25 * & 99.14 * \\ \hline
	\multicolumn{1}{r|}{}                                   & (v/ /*) & (0/8/0) & (0/0/8) & (0/1/7)    & (0/5/3) & (0/2/6) & (0/1/7) \\ \cline{2-8} 
	\end{tabular}
	\renewcommand{\tablename}{Tabla}
	\caption{\label{tabla1}Test Output: Analysis result}
\end{table}

\subsection{a) ¿Hay algún agente que parezca más adecuado?}
\p{Parece que el agente \textit{T3BotAgent} tiene un rendimiento mejor que el humano \textit{T3HumanAgent}, pero la diferencia
es irrisoria.}

\subsection{b) ¿Hay algún conjunto de datos particular que parezca más adecuado?}
\p{Parece ser que los mejores resultados se obtienen en los conjuntos \textit{noObs} y \textit{noObs\_noEva} con independencia
del algoritmo utilizado.}

\subsection{c) ¿Qué algoritmo parece más adecuado?}
\p{Puesto que en la mayoría de casos se ha empeorado respecto al caso base (utilizando J48), diríamos que el algoritmo más
adecuado es pues J48. Asimismo le sigue muy muy de cerca PART, solo habiendo empeorado mínimamente para el conjunto de datos
\textit{T3BotAgent\_Discr\_Selec}.}

\subsection{d) ¿Son los resultados del mejor algoritmo mucho mejores que los del resto?}
\p{Con salvedad de los resultados obtenidos con ZeroR, que son bastante malos en comparación al resto, los resultados obtenidos
con el mejor algoritmo (J48) no son mucho mejores que los del resto.}

\subsection{e) Cambiar el criterio de comparación y comparar los resultados. ¿Los resultados guardan relación con los
proporcionados en \textit{Percent\_correct}? ¿Qué otra métrica habéis seleccionado? ¿Por qué?}
\p{Tras probar varios criterios de comparación, podemos asegurar que aquellos que tienen que ver con el número o porcentaje
de instancias clasificadas correctamente o erróneamente guardan relación con los del comparador \textit{Percent\_correct},
ya que este se basa en el número de instancias clasificadas correctamente para comparar los resultados.}
\p{Entre los otros filtros que hemos probado se encuentra \textit{UserCPU\_Time\_testing}, ya que consideramos que una parte
muy imporante de un algoritmo de aprendizaje automático es el tiempo utilizado para clasificar las instancias y su complejidad
temporal. Hemos descubierto así que el algoritmo perezoso IbK es por norma más rápido que el resto.}

\subsection{f) Generar con el \textit{Explorer} de Weka los modelos que te parezcan más adecuados con los datos que
correspondan. ¿Cuales habéis generado? ¿Son estos modelos tan adecuados como parecían? ¿Por qué?}
\p{Hemos partido del modelo \textit{Selec} que se nos indicó que creáramos y hemos generado y evaluado los siguientes modelos:}
\begin{itemize}
	\item El modelo \textit{Selec} base
	\item Un modelo que no contiene la información observada a futuro
	\item Un modelo que no contiene ninguna observación de \texttt{mergeObservation()}
	\item Un modelo que combina los dos anteriores
\end{itemize}
\p{Tras la evaluación, hemos obtneido los siguientes resultados:}
\begin{table}[h]
	\centering
	\begin{tabular}{l|r|cccccc|}
	\hline
	\multicolumn{1}{|l|}{Dataset}                                & J48     & PART    & ZeroR   & NaiveBayes & IbK1    & IbK3    & IbK7    \\ \hline
	\multicolumn{1}{|l|}{T3BotAgent\_Discr\_Selec}               & 99.70   & 99.52   & 28.80 * & 98.67 *    & 99.35   & 98.99 * & 98.48 * \\
	\multicolumn{1}{|l|}{T3BotAgent\_Discr\_Selec\_noFut}        & 99.70   & 99.52   & 28.80 * & 99.45      & 99.44   & 98.90 * & 97.78 * \\
	\multicolumn{1}{|l|}{T3BotAgent\_Discr\_Selec\_noObs}        & 99.70   & 99.70   & 28.80 * & 98.79 *    & 99.66   & 99.55   & 98.91 * \\
	\multicolumn{1}{|l|}{T3BotAgent\_Discr\_Selec\_noObs\_noFut} & 99.70   & 99.70   & 28.80 * & 99.70      & 99.67   & 99.41   & 98.72 * \\ \hline
	\multicolumn{1}{r|}{}                                        & (v/ /*) & (0/4/0) & (0/0/4) & (0/2/2)    & (0/4/0) & (0/2/2) & (0/0/4) \\ \cline{2-8} 
	\end{tabular}
	\renewcommand{\tablename}{Tabla}
	\caption{Test Output 2: Analysis result}
\end{table}
\p{Como se puede observar, no mejora nada el análisis, aunque mejora el rendimiento de otros algoritmos (disminuyen frente a J48
pero menos) si se compara con los resultados obtenidos anteriormente, recogidos en la tabla 1 \nameref{tabla1}.}
\p{Creemos que estos modelos son adecuados porque obvian la posición X e Y de Mario, que aunque con ella se consigue un 100\% de
aciertos, es debido a que se está produciendo un sobreajuste porque el algoritmo está aprendiendo el mapa y no realmente a jugar.}

\subsection{g) Elegir un modelo final y justificar la respuesta.}
\p{Si tuviéramos que usar J48 y a la vista de los resultados arrojados, probablemente nos quedaríamos con el modelo que definimos
inicialmente, \textit{T3BotAgent\_Discr\_Selec}. Sin embargo, Nos gustaría tomar información adicional en las instancias de
entrenamiento.}
\p{No se ha incluido porque para este tutorial hemos seguido fielmente las exigencias del enunciado, incluyendo en la toma de
ejemplos solo los datos solicitados en el tutorial 1 y 3. Si de nosotros dependiera, tendríamos en cuenta también las teclas que
pulsa Mario o si este está en el suelo o en medio de un salto, por ejemplo, además de ciertas transformaciones extra con los datos
antes de escribir a los ejemplos.}

\subsection{h) ¿Por qué o para qué os parece adecuado el uso del \textit{Experimenter} de Weka?}
\p{El experimenter nos permite trabajar con varios set de datos distintos, así como con distintos algoritmos de aprendizaje
automático que nos ayuden a clasificar las instancias. Esto nos permite conocer qué sets son más representativos para el parendizaje
o para qué algoritmo de aprendizaje, así como qué algoritmo u algoritmos son los más útiles o eficientes dados nuestro set de datos.}

\end{document}
